\documentclass{article}

\usepackage{amsmath} % align*-environment
\usepackage{amssymb} % e.g. \mathbb{R}
\usepackage{amsthm} % newtheorem

\newtheorem{lemma}{Lemma}

\newcommand{\R}{\mathbb{R}}
\newcommand{\vctr}[1]{\mathbf{#1}}
\newcommand{\point}[1]{\mathbf{#1}}

\newcommand{\mat}[1]{\mathbf{#1}}
\newcommand{\pMat}[2]{\mat{P_{#1 \leftarrow #2}}}
\newcommand{\vMat}[2]{\mat{V_{#1 \leftarrow #2}}}

\newcommand{\colvec}[1]{\begin{pmatrix}#1\end{pmatrix}}

\DeclareMathOperator{\adj}{adj}

\begin{document}

\section{transforming normals}

\begin{lemma}
Let \(\mat{M} \in \R^{3\times 3}\) be an invertible matrix. And let \(\vctr{a}, \vctr{b} \in \R^3\). Then
\[
(\mat{M}\vctr{a})\times(\mat{M}\vctr{b}) = \adj(\mat{M}) (\vctr{a} \times \vctr{b})
\]
\end{lemma}
\begin{proof}
 ...
\end{proof}



\section{barycentric coordinates}

\[
\point{p} = \lambda_0 \point{p_0} + \lambda_1 \point{p_1} + \lambda_2 \point{p_2}
\]
where \(\lambda_0 + \lambda_1 + \lambda_2 = 1\).

Plugging in \(\lambda_0 = 1 - \lambda_1 - \lambda_2\):
\begin{align*}
%p &= (1 - \lambda_1 - \lambda_2) p_0 + \lambda_1 p_1 + \lambda_2 p_2 \\
\point{p} &= \point{p_0} - \lambda_1 \point{p_0} - \lambda_2 \point{p_0}  + \lambda_1 \point{p_1} + \lambda_2 \point{p_2} \\
&= \point{p_0}  + \lambda_1 (\point{p_1}-\point{p_0}) + \lambda_2 (\point{p_2} - \point{p_0})\\
&= \point{p_0} + \lambda_1 \point{\Delta p_1} + \lambda_2 \point{\Delta p_2}
\end{align*}

where
\begin{align*}
\point{\Delta p_1} &:= \point{p_1} - \point{p_0}\\
\point{\Delta p_2} &:= \point{p_2} - \point{p_0}
\end{align*}

Thus
\[
 \colvec{\point{p}\\1} = \pMat{oc}{bc} \colvec{\lambda_1 \\ \lambda_2 \\ 1}
\]
where
\[
 \pMat{oc}{bc} = \begin{pmatrix} 
                  \point{\Delta p_1} & \point{\Delta p_2} & \point{p_0} \\
                  0 & 0 & 1
                 \end{pmatrix}
\]


\end{document}
